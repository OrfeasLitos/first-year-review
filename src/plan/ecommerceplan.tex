\subsection{E-Commerce}
  Our model for reputation in e-commerce is still incomplete. As already mentioned, we
  have to prove that $\Pi_{\mathrm{Trade}}$ UC-realises $\mathcal{F}_{\mathrm{Trade}}$.
  This will not be hard to prove, but we have chosen to defer this proof to a later point
  in time because it is still likely that the details of both the protocol and the
  functionality will change.

  A more fundamental and pressing issue is the exact formulation of security definitions
  related to the model. In particular, it is unclear whether
  Definition~\ref{def:nocheatsec} is too strong or needs further refinements. Currently
  nothing stops Alice from deviating and falsely reporting that she has been
  \texttt{cheated} after an honest trade, apart from speculated game-theoretic incentives.
  This means that the current version of $\Pi_{\mathrm{SAT}}$ is not secure under this
  definition.

  On the other hand, we have seen that $\mathcal{F}_{\mathrm{SAT}}$ is cheating--secure.
  This is proof that $\Pi_{\mathrm{SAT}}$ does not UC-realise
  $\mathcal{F}_{\mathrm{SAT}}$. This is another important problem with the current
  formulation of the model. If we consider static corruptions, the Adversary can very
  easily break this kind of security by telling to $\mathcal{E}$ (truthfully or not) that
  she has been \texttt{cheated}, or by cheating on a trade and causing another player to
  report the cheat. Even if we ignore the Adversary, our intuition is that every
  cheating--secure protocol is either of little use --- consider a protocol where no
  trades ever take place, or a protocol where a buyer trades only with the single most
  trustworthy vendor --- or may actually achieve security by hiding the fact that cheats
  happen, e.g. in case the protocol never reports any cheats that took place. This
  obviously defeats the purpose of this security definition. We will consider two
  approaches to remedy this issue.

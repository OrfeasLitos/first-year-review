\subsection{Contigency Plan}
  There are several things that can go wrong with the aforementioned stages of further
  development. The most obvious is the danger of over-generalisation. It may be the case
  that the problems we have embarked to solve prove untractable. In this case, we will
  have to consider more specific simpler cases in which arguing will be easier. For
  example, it is possible that the evidently general way in which we attempt to model
  payment channels has too little structure, thus reducing our ability to express or prove
  interesting properties. In this case, we will focus on special cases under which we will
  be able to effectively argue with respect to these properties.

  Another possible issue in this rapidly evolving area is that our problems are solved by
  other researchers before we can provide a solution. We hope that this will not be a
  serious impediment, as our exploration of the field, in combination with said results
  from third parties, will provide us with material to reroute our research to other
  relevant open questions.

  We would like to note that having two different, albeit related problems at hand makes
  it easier to backtrack and devote more time to one of the two in case we cannot keep
  pursuing the other for any reason; we have effectively reduced the danger of a ``single
  point of failure''. Nevertheless, we hope to eventually manage to attain a unified
  thesis that will stem from the research of these two related issues.

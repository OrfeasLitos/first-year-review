\subsubsection{Connecting with UC modules and increasing robustness} \ \\*[0.5\baselineskip]
  The following issues are considered refinements to better approximate the real world and
  will not be resolved until we first provide a functional prototype of the simplified
  system that we have been discussing up to now.

  Further inflection on the exact nature of the global functionalities is needed. More
  specifically, the interface with $\mathcal{G}_{\mathrm{Ledger}}$ that we currently use
  is not aligned with that of the UC formalisation of the functionality~\cite{ucledger}.

  As far as robustness goes, we would like to expand the current model to additionally
  accomodate for digital assets that can be placed on a blockchain --- thus making it
  possible to achieve consensus on who is the rightful owner. It seems natural to treat
  this kind of assets differently, since less trust needs to be implicated for their
  handling.

  Additionally, in our current model there is no provision for a vendor who delivers an
  asset of inferior quality or a different asset from the one promised; we just account
  for trades that either complete successfully or do not complete at all. Since this kind
  of behaviour is common and is a very usual source of disputes, our model should be able
  to handle this situation in a satisfactory manner.

  Furthermore, we would like to present a formal treatment on whether it is possible to
  realise $\mathcal{G}_{\mathrm{Assets}}$ or not and if so, how. The point of the matter
  is that currently $\mathcal{G}_{\mathrm{Assets}}$ keeps track of who owns which asset,
  but we would like to avoid relying on such a centralised repository of titles in the
  real world. To make matters worse, since physical assets cannot be reliably put on a
  blockchain, the real-world equivalent of this functionality is highly non-trivial and
  certainly needs some form of trust between players.

\subsubsection{Security definitions and the role of the Adversary} \ \\*[0.5\baselineskip]
  A more fundamental and pressing issue is the exact formulation of security definitions
  related to the model. In particular, it is unclear whether
  Definition~\ref{def:nocheatsec} is too strong or needs further refinements. Currently
  nothing stops Alice from deviating and falsely reporting that she has been
  \texttt{cheated} after an honest trade, apart from speculated game-theoretic incentives.
  This means that the current version of $\Pi_{\mathrm{SAT}}$ is not secure under this
  definition.

  On the other hand, we have seen that $\mathcal{F}_{\mathrm{SAT}}$ is cheating--secure.
  This is proof that $\Pi_{\mathrm{SAT}}$ does not UC-realise
  $\mathcal{F}_{\mathrm{SAT}}$. This is another important problem with the current
  formulation of the model. If we consider static corruptions, the Adversary can very
  easily break this kind of security by telling to $\mathcal{E}$ (truthfully or not) that
  she has been \texttt{cheated}, or by cheating on a trade and causing another player to
  report the cheat. Even if we ignore the Adversary, our intuition is that every
  cheating--secure protocol is either of little use --- consider a protocol where no
  trades ever take place, or a protocol where a buyer trades only with the single most
  trustworthy vendor --- or may actually achieve security by hiding the fact that cheats
  happen, e.g. in case the protocol never reports any cheats that took place. This
  obviously defeats the purpose of this security definition. We will consider two
  approaches to remedy this issue.

  The first is to follow the paradigm set forth in~\cite{rationalprotocol}, according to
  which we can assign a utility function to the Adversary to model the cost of corruptions
  and her benefit when security is broken. We can then argue that a specific
  $\Pi_{\mathrm{SAT}}$ protocol is not secure, but is \textit{optimal} with respect to
  some (or ideally all) utility functions of the Adversary --- meaning that there exists
  no protocol with which the Adversary may obtain higher utility. This approach gives us
  greater flexibility and allows us to attain useful results even if we do not achieve
  security.

  Secondly, we can add more security definitions and try to create a protocol that
  achieves them. For example, it is worthwile to look for a security definition that, if
  achieved, ensures that no false reports of cheated trades take place, or that no
  unresolvable conflicts take place.

  A related issue that is still not tackled in its entirety is the role of the Adversary.
  The current setting poses a significant departure from the usual cryptographic one in
  the sense that the participants' are not as clear-cut as e.g. in the case of modelling
  encryption. In our case, it may be to the participants' advantage to commit subtle
  variations from the protocol but at the same time still want to maintain its security
  guarantees. In this sense, arguing with respect to every polynomial algorithm may be an
  overkill. Most probably though, a combination of both a conventional Adversary and
  rational actors will be needed. The former could model a state-level actor determined to
  undermine the entire scheme at all costs, whereas the latter would represent small-scale
  players that want to enjoy the security properties of the system while maximising their
  own gain, possibly creating a ``tragedy of the commons'' effect~\cite{tragedy} in the
  process. Combining the two actors in one system is something that has not been
  accomplished yet to a satisfactory degree to our knowledge and indeed seems too daunting
  an undertaking.  Alternative approaches such as~\cite{rationalprotocol} must be explored
  and a sensible reach for the capacities of the Adversary must be established.

  We would like to have solved the issues of security definitions, define the capabilities
  of the Adversary and, if suitable, to employ the rational protocol design approach to a
  satisfactory degree within the next two to three months, i.e. until September 2018. A
  possible publishable outcome of this stage would be to prove that, under a particular
  utility function for the Adversary and with respect to some plausible security
  definitions, a certain protocol is optimal in the sense explained previously.

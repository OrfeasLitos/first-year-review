\subsubsection{Further considerations} \ \\*[0.5\baselineskip]
  To conclude this subsection, we would like to lay out some important --- if rather
  philosophical --- observations that have arisen through inflection on the topic at hand
  and should be kept in mind to guide the further development of the project.

  We have already alluded to the fact that there exist scenarios in which consensus is not
  achievable. In contrast to systems that exist entirely on blockchains, our model aspires
  to facilitate trades that include physical assets. There exist several fundamental
  differences between those and on-chain assets. Physical assets cannot be created out of
  nothing because of physical, immutable laws (e.g. the conservation of mass); this
  does not hold for digital assets, which may indeed be cloneable: there exist smart
  contracts where it is legal for new assets to emerge from nothing~\cite{cryptokitties}.
  Given the flexibility of the digital world, we can create protocols where consensus on
  who owns which asset and exactly how history transpired can be reached, based on modest
  (if disputable) assumptions such as honest majority. The physical world does not possess
  such infrastructure, so simple and complex scenarios where consensus is unreachable
  abound. For example, Alice may hand over an asset to Bob in a private room and later ask
  it back, only to have Bob claim that she gave it to him as a present. Any external
  arbiter will never be able to conclusively decide who is right, thus consensus is
  unreachable.

  The simplistic approach of putting everything on the blockchain is not sufficient for
  two reasons: Firstly, it would be prohibitively expensive to put videos of everything
  on-chain. Secondly, even if we somehow managed to scale blockchains to accommodate for
  such needs, just access control rules to such material are not clear at all and would
  probably constitute a science in and of itself. Last but not least, interpreting this
  information in case of dispute is no trivial task. Current consensus protocols
  accommodate a strictly defined set of assertions and do not permit contradictory facts
  to be recorded in the first place. In the physical world however, disagreements,
  disputes and inconsistencies are definitely possible and we have to constantly work
  around them. A standard approach taken is escalation to a third party chosen to be as
  impartial as possible, commonly taking the form of a court of law. When however accounts
  mismatch, it often devolves to a ``my word against yours'' situation where the
  apparently most trustworthy account is deemed true. This is where trust comes
  unavoidably in action. This is the kind of trust we are trying to harness with the
  current work.

  The ultimate target of this research is to create a plausible framework within which we
  can argue on the trustworthiness of institutions, people and statements based on one's
  experience, knowledge and interconnections in order to decide which course of future
  action will maximise some particular objective. Having such a framework in place, we
  could then recognise general tendencies and properties of the system as a whole when
  each individual agent tries to attain their objective. Ideally we will be able to set up
  the game in a particular fashion so as to achieve desirable properties for the system,
  such as maximisation of social welfare.

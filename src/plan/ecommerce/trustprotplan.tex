\subsubsection{Trust functionality and protocol} \ \\*[0.5\baselineskip]
  An additional question that remains as of yet unanswered is the way in which
  $\Pi_{\mathrm{Trust}}$ will be designed. Apart from placeholder and easily attackable
  solutions (e.g. choose the vendor with the most reported honest trades), the only real
  candidate is a suitable wrapper around ``Trust is Risk''~\cite{trustisrisk}. For
  reference, ``Trust is Risk'' is a reputation system in which Alice chooses to ``directly
  trust'' some funds to other players, publicly acknowledging that they can easily steal
  those funds from her. This creates a trust network, which Alice can leverage to
  calculate her ``indirect trust'' towards a potentially unknown player Bob as the maximum
  flow~\cite{maxflownm} from herself to Bob. This number is a measure of her trust towards
  him, built in a decentralised manner.

  We speculate that we can effectively use this system to build a $\Pi_{\mathrm{Trust}}$
  with desirable properties. The construction will roughly be as follows: Upon receiving a
  desire and a list of potential vendors from Alice, $\Pi_{\mathrm{Trust}}$ returns the
  vendor that Alice indirectly trusts the most. Upon receiving a report that Alice has
  been \texttt{cheated} by Bob for a loss of $p_1$ coins, $\Pi_{\mathrm{Trust}}$ decreases
  Alice's direct trusts in a way that her indirect trust towards Bob is reduced by $p_1$.
  Upon receiving a report that Alice has successfully completed a trade with Bob, during
  which she was exposed to a risk of losing $p_2$ to Bob, $\Pi_{\mathrm{Trust}}$ increases
  Alice's direct trust towards Bob by $p_2$.

  This rough protocol is intuitively appealing because we have not come up with obvious
  attacks and it contains a minimal number of hand-tuned parameters. Furthermore, it seems
  to align correctly the incentives of the various players, be it buyers, vendors or
  middlemen. A buyer wants to increase her trust towards a vendor of honourable past and
  penalise a dishonest vendor, thus a vendor is incentivised to refrain from cheating.
  What is more, from the point of view of a middleman, Alice would like to only directly
  trust players with ``good taste'', so that she could build a name for ``good taste''
  herself and acquire direct (and thus indirect) trusts from others that share the same
  taste. She can even charge a fee for those leveraging her service in providing
  trustworthy references. Thus she is incentivised to choose wisely who she directly
  trusts based both on how good vendors they are and on who they in turn directly trust.

  Last but not least, calculating indirect trust based on the maximum flow has a
  sociological justification~\cite{jgs}, as people tend to lend money more commonly to
  those towards which their maximum flow is the highest, as calculated on a network where
  edges represent the amount of time two neighbouring parties have spent together.

  Our target for the following three months is to formally describe the above sketch,
  decide upon its details and argue whether it satisfies our security definitions and, if
  so, under which conditions.

\subsubsection{$\mathcal{F}_{\mathrm{Trust}}$} \ \\*[0.5\baselineskip]
  This last functionality is probably the most crucial --- it is certainly the main
  missing component in order to model a functional, secure marketplace. The Trust
  functionality is responsible for choosing a suitable vendor to satisfy a particular
  desire. Going in more detail, it receives a desire and a list of possible vendors from
  Alice and returns a single vendor that is the most trustworthy for Alice to buy from.

  There are several important things to note here. First of all, in contrast to a real
  trusted third party (e.g. ebay), this functionality cannot be ``held accountable'' if it
  recommends a bad vendor. No legal action can be taken against it. Contrary to a
  real-world enterprise, it does not have any incentives, it just follows its
  specification; the mechanism it uses to choose suitable vendors is common knowledge to
  all players, thus security through obscurity (as employed by real-world enterprises that
  closely guard their reputation system mechanism) is out of the question. An advantage
  that this functionality has is that we can choose to provide it with oracle access to
  the utilities of all players; this is however a luxury that is necessarily absent from
  any protocol that will attempt to realise $\mathcal{F}_{\mathrm{Trust}}$. Last but not
  least, in the real world players have the option to employ a variety of fundamentally
  different mechanisms in order to choose a suitable vendor, with some being word of
  mouth, search engines, rating websites and the reputation systems of online
  marketplaces. $\mathcal{F}_{\mathrm{Trust}}$ should ideally encompass the possibilities
  of all these methods and more. Only then will we be able to safely assume that all
  coalitions of players will have incentives to keep using $\mathcal{F}_{\mathrm{Trust}}$,
  instead of employing a different method.

  As of now, we have built an $\mathcal{F}_{\mathrm{Trust}}$ that, having oracle access to
  the utilities of all players, chooses a vendor in such a way that:
  \begin{itemize}
    \item The vendor finds it beneficiary not to cheat.
    \item Amongst all possible vendors that conform to the previous constraint, the
    additional utility the buyer will enjoy after the trade is maximised.
  \end{itemize}

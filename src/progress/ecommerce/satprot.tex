\subsubsection{$\Pi_{\mathrm{SAT}}$} \ \\*[0.5\baselineskip]
  An overview of this protocol has already been provided. Here we go into further detail.
  Let Alice be a player that executes an instance of the protocol. First of all, we
  clarify that Alice will not do any operation upon receiving any messages if she has not
  already received her utility function from $\mathcal{E}$. Apart from message containing
  the utility function, there are three more types of messages that $\Pi_{\mathrm{SAT}}$
  can receive.

  Upon receiving \texttt{(satisfy, d, L)} from $\mathcal{E}$, Alice asks
  $\mathcal{F}_{\mathrm{Trust}}$ to choose the best vendor from the list \texttt{L}. She
  then asks the chosen vendor if he can satisfy her desire, who responds with a suitable
  asset and a price if he wants to trade. If the utility function's value would increase
  after this trade, Alice instructs $\mathcal{F}_{\mathrm{Trade}}$ to proceed with the
  agreed transaction. She then waits for the result of the trade, which can be either
  \texttt{satisfied} or \texttt{cheated}. In either case, Alice informs both
  $\mathcal{F}_{\mathrm{Trust}}$ and $\mathcal{E}$ with a relevant message and goes idle.

  If the vendor does not want to trade or if the proposed trade is not favourable for
  Alice, she asks $\mathcal{F}_{\mathrm{Trust}}$ for a new vendor. This process is
  repeated until a suitable vendor is found or until all vendors from the list \texttt{L}
  are exhausted. In the latter case, Alice informs $\mathcal{E}$ that the desire remained
  \texttt{unsatisfied}.

  As a vendor, Alice may be contacted by Bob asking her if she can satisfy a desire of
  his. If her utility function's value would increase after an honest trade (or after
  cheating Bob) she provides Bob with a specific offer. The last message Alice reacts upon
  is a message from $\mathcal{F}_{\mathrm{Trade}}$, asking her whether to cheat on an
  already initiated trade or not. She again decides based on the amount of money she has
  been paid, the value of the asset to trade in and the cost to her reputation. Note that
  it is as of yet unclear how Alice measures the cost that specific choices will have upon
  her reputation. It seems to be inextricably tied to the specific way
  $\Pi_{\mathrm{Trust}}$ is specified. We should also clarify that every player can act
  both as a buyer and a vendor interchangeably.

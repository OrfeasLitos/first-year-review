\subsubsection{Security Definitions} \ \\*[0.5\baselineskip]
  Apart from defining ideal functionalities and real-world protocols, it is useful to give
  some security definitions. Until now we have given only one definition.

  \begin{definition}[Security against Cheating]
  \label{def:nocheatsec}
    We say that a protocol is secure against cheating if $\forall \: \mathrm{PPT} \:
    \mathcal{E}, \Pr\left[\mathcal{E} \: \mathrm{receiving} \: \left(\mathtt{cheated, \_,
    \_}\right)\right]$ is negligible.
  \end{definition}
  \noindent Here a function $f$ is negligible if $\forall c \in \mathbb{R}, \exists n_0
  \in \mathbb{N}: \forall n \geq n_0, f\left(n\right) \leq \frac{1}{n^c}$.

  Since $\mathcal{F}_{\mathrm{SAT}}$ never cheats, it obviously is secure against
  cheating. More definitions will be needed however, because it is very simple to create a
  corresponding $\Pi_{\mathrm{SAT}}$ protocol that is secure against cheating: simply
  never trade anything. Also this security definition does not protect against maliciously
  sent $(\mathtt{cheated}, \_, \_)$ messages.

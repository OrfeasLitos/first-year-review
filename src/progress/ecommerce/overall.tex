\subsubsection{Overall game description} \ \\*[0.5\baselineskip]
  After providing the utility functions and the endowments, the main part of the game
  begins. $\mathcal{E}$ can send a message to $\mathrm{Alice} \in \mathcal{P}$ that
  contains a desire $d$ to be satisfied. To satisfy this desire, Alice has to buy an asset
  that satisfies $d$ from some other player. The first and most important task Alice faces
  is who to trade with. For that purpose, she consults with the functionality
  $\mathcal{F}_{\mathrm{Trust}}$, which hopefully responds with a trustworthy vendor, say
  Bob. Then Alice asks Bob whether he is willing to satisfy her desire with a satisfying
  asset and, if so, at what price. Given that Bob offers a reasonable price, Alice
  instructs the functionality $\mathcal{F}_{\mathrm{Trade}}$ to pay Bob the designated
  price in exchange for the satisfying asset. The functionality pays Bob and crucially
  asks him if he wants to complete the exchange honestly or cheat. It then completes the
  trade as instructed by Bob by interacting with the two global functionalities
  appropriately and reports back to Alice the result. Finally Alice reports the result to
  $\mathcal{E}$ and $\mathcal{F}_{\mathrm{Trust}}$. The latter updates the trust
  properties connected to Bob depending on Alice's report. Just like Bob could cheat and
  not deliver the satisfying asset, Alice has the ability to misreport her experience with
  Bob.

  The fact that the buyer pays for the asset before the seller delivers it is a design
  choice; we could have likewise chosen that the seller should first deliver and then
  expect payment. We chose this ordering however to closer mimic the sequence of events in
  most real-world exchanges.

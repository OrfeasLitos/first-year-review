\subsubsection{Utility functions properties} \ \\*[0.5\baselineskip]
  There is no game theoretic analysis without utility functions. In our case, each player
  can act upon receiving a message only if she has been given a utility function by
  $\mathcal{E}$. Any such utility function is defined for every moment in time, every
  possible asset ownership and every amount of coins owned:
  \begin{equation*}
    u_{\mathrm{Alice}} : \mathrm{Time} \times \mathrm{Assets} \times \mathrm{Money}
    \rightarrow \mathbb{R} \enspace.
  \end{equation*}
  Some properties that we may assume for the utility functions if needed are
  quasi-concavity, continuity and non-satiation~\cite{marketequilibrium}, at least
  regarding a restriction of the utility function for a particular moment in time.

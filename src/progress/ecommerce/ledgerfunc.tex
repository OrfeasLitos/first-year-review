\subsubsection{$\mathcal{G}_{\mathrm{Ledger}}$} \ \\*[0.5\baselineskip]
  This functionality keeps track of the digital assets, i.e. coins. For the purposes of
  this work, all interactions are similar to those of $\mathcal{F}_{\mathrm{Assets}}$,
  except for the fact that Alice can prove ownership of some funds to an arbiter. This
  reflects the fact that, through the use of blockchains, Alice can prove that she owns a
  particular key that has the ability to spend a particular amount of coins.

  This difference is fundamental. Blockchains allow for consensus on who owns which coins
  because the whole history is distributed and immutable. This is not the case for
  physical assets. Conflicting accounts of past events can happen in such a way that no
  third party can tell which version is true; crucial events may be hidden (on purpose or
  because nobody observed them) and consensus may be unreachable. Furthermore, it seems
  impossible to record the state of the entire physical world on the blockchain without
  trusting a third party. It is our intent to capture this tension between digital and
  physical assets and thus we employ two separate global functionalities.

\subsubsection{The ideal market functionality: $\mathcal{F}_{\mathrm{SAT}}$} \
\\*[0.5\baselineskip]
  Following the simulation-based security paradigm and in order to help us conceptualise
  how an ideal market should function, we have additionally designed a satisfaction
  functionality $\mathcal{F}_{\mathrm{SAT}}$. In the world where this functionality
  exists, the protocol players are dummy, i.e. they simply forward all messages they
  receive from $\mathcal{E}$ to $\mathcal{F}_{\mathrm{SAT}}$ and vice versa. Therefore,
  $\mathcal{F}_{\mathrm{SAT}}$ knows the utilities of all players and can complete optimal
  trades without ever having to show trust to other actors.

  More precisely, the current version of $\mathcal{F}_{\mathrm{SAT}}$ behaves as follows:
  upon receiving a desire from Alice, it creates a list of vendors, assets and prices that
  constitute possible trades, each of which would be beneficial both to Alice and the
  implicated vendor. It then sends this list to the Adversary $\mathcal{A}$; the Adversary
  sends back the trade that will take place and $\mathcal{F}_{\mathrm{SAT}}$ executes it.
  If the Adversary does not respond with a valid message, then
  $\mathcal{F}_{\mathrm{SAT}}$ chooses the best trade for Alice and executes that. Note
  that this is the first time we have implicated the Adversary in any way.

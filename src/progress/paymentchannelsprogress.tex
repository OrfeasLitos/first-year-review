\subsection{Research on Payment Channels}
  In this part of the research we have mainly focused on understanding the general
  framework within which the various realisations of payment channels can be classified,
  compared and evaluated. To achieve this, we reviewed the relevant literature, which has
  rapidly grown to be quite extensive and contains both concrete
  systems~\cite{abunchofthings} and attempts for the formalisation of payment
  channels~\cite{statechannels}. Section~\ref{sec:literaturereview} provides a summary of
  the field.

  The inquiry into the nature of payment channels amounted to the following (possibly
  incomplete) list of characteristics any payment channel has:
  \begin{itemize}
    \item Number of participants in the channel
    \item Relevant on-chain transactions (that keep funds locked in the channel)
    \item Available actions, e.g. open, update to new state, execute rules on chain, close
    \item Who needs to sign for each action, who is notified, how many rounds of
    communication does this amount to?
    \item What information leaks to the blockchain?
    \item Under what circumstances an operation cannot complete? (e.g. concurrency issues,
    veto power with misaligned incentives)
    \item Which participants know the identity of which participants?
    \item Is there an upper bound to the amount of updates? How is this number decided?
    \item Can a participant unilaterally commit on-chain?
    \item Up to how much money can a participant unilaterally obtain?
    \item What can a malicious party do? If it corrupts more participants it can do more?
    \item Can a malicious/honest-but-curious party that is a participant learn who is
    transacting with who, especially in payments she is not involved?
    \item How much slower is the process in case of a malicious party?
    \item How expensive are the actions? (CPU, memory, storage)
    \item How expensive are interactions with the blockchain? (fees, time, etc.)
  \end{itemize}

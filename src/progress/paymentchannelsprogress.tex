\subsection{Research on Payment Channels}
  In this part of the research we have mainly focused on understanding the general
  framework within which the various realisations of payment channels can be classified,
  compared and evaluated. To achieve this, we reviewed the relevant literature, which has
  rapidly grown to be quite extensive and contains concrete
  systems~\cite{lightning,perun,teechan,duplexchannels,anonchannels,spriteschannels,fulgorrayo,bolt},
  applications on top of channels~\cite{blindsigncoins}, ideas towards inter-blockchain
  transactions~\cite{atomicswaps} and attempts for the formalisation
  of payment channels~\cite{statechannels}. Subsection~\ref{sec:literaturereview:channels}
  provides a summary of the field.

  We will give here an informal description of the simplest use case of a payment channel.
  In this scenario, Alice and Bob are about to enter a financial relationship that will
  last for some time and during which Bob will have to pay Alice small sums very often.
  Performing each transaction on-chain would be unwieldy both because of the
  per-transaction fees lost to miners/minters and because of the need for Alice to wait
  for confirmation after each transaction.

  The solution is for Bob to initially publish on-chain a transaction with e.g. 10,000
  coins that can be spent only by an ``update'' transaction $u_0$ that would deposit the
  whole sum to an address controlled by Bob. He privately sends $u_0$ to Alice. Note that
  Alice cannot make $u_0$ herself, because it is signed by Bob. If Bob wants to pay Alice
  e.g 1 coin, he creates and privately sends her a new update transaction $u_1$ which, if
  published, would give 9,999 coins to Bob and 1 coin to Alice. He also sends her a
  ``revocation'' transaction $r_0$ that would invalidate $u_0$ if it was to be published.
  Therefore, Alice is sure that Bob cannot take this coin back as long as she watches the
  blockchain. The same idea can be applied more times, until all 10,000 coins are handed
  over to Alice or either party decides to publish the latest $u_i$ and unlock both
  parties' funds.

  This simplified description has several shortcomings, but it provides intuition on how
  many transactions can happen off-chain and still require no trust between the parties;
  both sides can unilaterally publish the latest update and get on-chain exactly the funds
  they are entitled to.

  As we will see in more detail in the literature review, there are much more capabilities
  to payment channels. Bidirectional channels, transfer of funds from one channel to
  another, privacy, implementation of arbitrary smart contracts and even exhange between
  different blockchains can happen off-chain. It is indeed very interesting to see how far
  these ideas can go and if they provide the much-needed solution to the blockchain
  scalability problem.

  The inquiry into the nature of payment channels amounted to the following (possibly
  incomplete) list of characteristics any payment channel has:
  \begin{itemize}
    \item Number of participants in the channel
    \item Directionality of the channel
    \item Relevant on-chain transactions (that keep funds locked in the channel)
    \item Available actions, e.g. open, update to new state, execute rules on chain, close
    \item Who needs to sign for each action, who is notified, how many rounds of
    communication does this amount to?
    \item How much time does a party have to wait between actions or between the various
    steps of each action?
    \item How often do parties need to check the blockchain for changes to the channel?
    \item What information leaks to the blockchain?
    \item What can be told on the automatic routing of payments? To what extent can it be
    non-blocking and private?
    \item Under what circumstances an operation cannot complete? (e.g. concurrency issues,
    veto power with misaligned incentives)
    \item Which participants know the identity of which participants?
    \item Is there an upper bound to the amount of updates? How is this number decided?
    \item What are the necessary tools the blockchain should provide?
    \item Can a participant unilaterally commit on-chain?
    \item Up to how much money can a participant unilaterally obtain?
    \item What can a malicious party do? If it corrupts more participants it can do more?
    \item Can a malicious/honest-but-curious party that is a participant learn who is
    transacting with who, especially in payments she is not involved?
    \item How much slower is the process in case of a malicious party?
    \item Can settling on-chain happen faster in case everybody is honest?
    \item How faster/cheaper can the system be if we assume some form of trust?
    \item Can arbitrary smart contracts be implemented off-chain? At what cost?
    \item How expensive are the actions? (CPU, memory, storage)
    \item How expensive are interactions with the blockchain? (fees, time, etc.)
  \end{itemize}

  An additional interesting question is how the design of the blockchain itself can
  facilitate the creation of such payment channels and whether such design choices would
  degrade the quality of the blockchain itself.

  \subimport{./progress/paymentchannels/}{model.tex}

\subsection{Modelling Decentralised E-Commerce}
  The main aim of this research is to construct a model that closely follows the dynamics
  of a marketplace. Following the Universal Composability framework, we define $n$
  interactive Turing Machines (ITMs) that represent the players that take part in the
  marketplace game. Their description is $\Pi_{\mathrm{SAT}}$, the satisfaction protocol.
  These ITMs comprise the set $\mathcal{P}$. Furthermore, we define an ITM named
  Environment and represented by $\mathcal{E}$.

  The latter ITM is the first to start functioning. Initially, its responsibility is to
  allocate a utility function to each player, drawn from a distribution on legal utility
  functions that is common knowledge \cite{knowledge} to the players. We will elaborate on
  the nature of the utility functions later on. Additionally, $\mathcal{E}$ provides
  players with an initial ``endowment'' of assets and money. Contrary to the utility
  functions, these are not sent only to the players, but additionally to the
  $\mathcal{G}_{\mathrm{Assets}}$ and $\mathcal{G}_{\mathrm{Ledger}}$ global
  functionalities respectively. These two functionalities exist with the sole purpose of
  keeping track of physical and digital assets respectively. We will later provide further
  explanation as to why we introduce these global functionalities.

  The current state of this part of the research can be found in
  \url{https://github.com/OrfeasLitos/UC-Trust/blob/master/what_is_trust.pdf}. We will
  now go into further detail regarding the construction and its parts.

  \subimport{./progress/ecommerce/}{overall.tex}
  \subimport{./progress/ecommerce/}{assetsfunc.tex}
  \subimport{./progress/ecommerce/}{ledgerfunc.tex}
  \subimport{./progress/ecommerce/}{tradefunc.tex}
  \subimport{./progress/ecommerce/}{trustfunc.tex}
  \subimport{./progress/ecommerce/}{utilityfunc.tex}
  \subimport{./progress/ecommerce/}{satprot.tex}
  \subimport{./progress/ecommerce/}{satfunc.tex}
  \subimport{./progress/ecommerce/}{combination.tex}
  \subimport{./progress/ecommerce/}{results.tex}
  \subimport{./progress/ecommerce/}{secdefn.tex}

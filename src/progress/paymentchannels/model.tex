\subsubsection{Model for Payment Channels} \ \\*[0.5\baselineskip]
  We have been considering the following simple but general model for payment channels.
  Stricter definitions, implications and realisation of existing solutions on this model
  will follow as further work.

  A payment channel is a tuple $PC \in \mathcal{PC}$ such that
  \begin{equation*}
    PC = \left(\left\{\left(P_1, c_1\right), \dots, \left(P_n, c_n\right)\right\},
    \left\{\left(e_1, b_1\right), \dots, \left(e_m, b_m\right)\right\}, f : \mathcal{A}^n
    \rightarrow \mathcal{PC}\right)
  \end{equation*}
  where $\sum\limits_{i = 1}^n c_i \leq \sum\limits_{i = 1}^m b_i$.

  $\left(P_i, c_i\right)$ represents the $i$-th player and her available funds on
  settling.

  $\left(e_j, b_j\right)$ represents the $j$-th on-chain endpoint and the corresponding
  funds that will be released for use in the blockchain if this endpoint is settled.

  $f$ is a function from player actions to a new payment channel. The new payment channel
  must have at most as much funds as the old.

  We understand that the current formalisation of payment channels contains a cyclic
  definition that stems from the way the function $f$ is used and we will strive to
  correct this problem before using this definition. We however want to maintain the
  intuition that each payment channel is accompanied by a transition function that defines
  the result of the various possible actions as new payment channels.

\subsubsection{Game Theory and Mechanism Design} \ \\*[0.5\baselineskip]
  There are several concepts relevant to reputation and trust in the realm of Game Theory.
  To begin with, the Vickrey-Clarkes-Grove mechanism~\cite{nisan2007algorithmic} is
  strategy-proof mechanism to select a particular outcome out of many in a way that
  maximises utilitarian social welfare and incentivises players to be honest about their
  preferences. Furthermore, there exist positive results on Nash Equilibria in games of
  fair resource allocation~\cite{cakecutting}. Such results may be leveraged to employ
  trustworthy mechanisms with envy-free outcomes --- two caveats are the finite horizon
  of these mechanisms and the possibly exponential resources needed to calculate outcomes.
  Trust to unknown agents is also related to the belief system used in bayesian
  games~\cite{bayesiangames}. These games are better suited to our case, since they can
  have an infinite horizon and the limits of each player's knowledge about the rest are
  more clearly defined.

  The study of economic trust rests heavily upon results regarding the existence of market
  equilibria~\cite{marketequilibrium} and the non-existence of fair and free elections
  under general circumstances~\cite{arrowtheorem}. These results show us the importance of
  a common understanding of the concept of value amongst players in order to achieve fair
  outcomes, a concept that to some extent can be identified with that of money.

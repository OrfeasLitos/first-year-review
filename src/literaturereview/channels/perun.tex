\subsubsection{Perun} \ \\*[0.5\baselineskip]
  Perun~\cite{perun} is a type of trustless payment channel designed for Turing-complete
  smart contract scripting languages. It has been implemented for Ethereum. Its main
  contribution is \textit{multistate channels} that allow the dynamic deployment of
  virtual contracts, known as \textit{nanocontracts}. Contracts of this type do not have
  to exist in the blockchain if all parties are cooperative and are only broadcast to the
  blockchain in case of a dispute.

  The paper describes specifically the use of such multistate channels for creating
  virtual payment channels between parties that do not have a basic payment channel
  between them, but both have basic multistate channels with an intermediary. Then the
  intermediary could substitute for the blockchain and thus a virtual payment channel on
  top of the two basic multistate channels can be created. The parties need the
  intermediary only for setting up the channel and to close it fast. If the intermediary
  refuses to close the channel, they can always fall back to the blockchain in order to
  close it.

  Perun is more robust and feature-rich than the Lightning Network. As already mentioned,
  arbitrary contracts can be implemented off-chain in the former, whereas the latter does
  not offer such functionality. Furthermore, persistent multi-hop channels can be built in
  Perun. Thus updating the state of long chains is independent of the length of the
  channel, in expense of the cost of setting up and closing the channel in case of
  dispute.

  One minor drawback in Perun is that a player that broadcasts an earlier state of the
  contract does not risk losing her rightful share of the funds (according to the latest
  state). This incentivizes players to broadcast earlier states of a contract more easily.

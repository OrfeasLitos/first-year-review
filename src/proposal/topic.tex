\section{Research Topic}
  The advent of bitcoin~\cite{bitcoin} has promised to revolutionise the way online
  commerce takes place, only to meet multiple hurdles in the way. Today there exists an
  entire ecosystem of cryptocurrencies, e.g.~\cite{ethereum,zcash,ouroboros}, each with is
  own benefits and drawbacks. Nevertheless, there are several fundamental questions that
  still remain unanswered. Some of the prominent issues are whether cryptocurrencies can
  become the global unit of account, whether it is possible to regulate and fairly
  distribute them, what are the exact benefits they provide when compared to fiat
  currencies, in which respects the latter outperform them and if it is possible to
  completely replace fiat currencies with decentralised cryptocurrencies for which the
  traditional measures of law enforcement are less applicable due to the global nature of
  blockchains.

  This research proposal is crucially motivated by the following observation: The single
  most advertised advantage of permissionless blockchains is the lack of need for trust
  between the interacting individuals \textit{and} the lack of a mediating trusted third
  party~\cite{bitcoin}. Unfortunately, this benefit is limited to a certain type of
  digital assets that admit a cryptographically secure proof of ownership and can be
  provably transferred to another party. Additionally, the overhead for setting such a
  system in motion is too big for the entire economy to be thus realised.

  \subimport{./proposal/}{ecommerce.tex}
  \subimport{./proposal/}{off-chain.tex}

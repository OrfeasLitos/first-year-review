\subsection{Game Theoretic approach}
  Before the advent of blockchains, most cryptographic protocols solved problems where
  there were clear-cut lines between the aims of honest parties and the machinations of
  the Adversary. For example, in the case of public key encryption~\cite{dh}, the honest
  parties should be able to exchange messages encrypted in a fashion that only those who
  know the private key can decrypt, even though the adversary has the ability to employ
  any method conceivable (within some computational limits) to read the original contents
  of the message. Any protocol that claims to realise secure encryption should satisfy
  (the formal version of) this requirement. It is intuitively obvious that it is in the
  benefit of the honest parties to follow such a protocol.

  On the other hand, blockchains made the state of affairs much more complex. Indeed,
  bitcoin was a concrete solution to an ill-defined problem; several years passed before a
  formal definition of the problem blockchains solve was formulated~\cite{backbone}.
  Nevertheless, this definition does not necessarily coincide with the desires of the
  participants to the protocol. The definitions demand that blockchains realise immutable
  ledgers that are inexorably extended, however the participants' goals span a great
  variety, such as maximising their share of coins or preventing certain transactions from
  entering the blockchain.

  This is where rational analysis comes into play.

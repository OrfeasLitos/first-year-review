\subsection{Off--chain payment channels}
  One additional impediment to the more widespread use of decentralised blockchains is the
  issue of scalability. Decentralised consensus protocols based on
  Proof-of-Work~\cite{hashcash} or Proof-of-Stake~\cite{ouroboros} make the assumption
  that the majority of the hashing power or of the stake respectively is controlled by
  honest parties, each of which processes every single transaction. To oversimplify, one
  can think that such protocols function only as long as every member of a crucial portion
  of their users processes all transactions. A great duplication of effort takes place in
  order to avoid trusting a particular set of users. It seems intuitively obvious that, no
  matter how much the parameters are tuned, such systems can only handle a limited amount
  of transactions per second and thus cannot compete traditional centralised solutions
  such as VISA in their current state~\cite{tps}.

  One solution to this issue is to ensure that most payments happen off--chain. A variety
  of mechanisms (discussed in more detail in Section~\ref{sec:literaturereview}) enable
  cryptocurrency owners to temporarily lock some funds in specially crafted channels with
  one on--chain transaction and subsequently perform any number of transactions between
  them with these funds without touching the blockchain. One more on--chain transaction
  unlocks the funds, ensuring each participant receives exactly the funds they owned in
  the latest state of the channel. Surprisingly, no trust is needed between the parties.
  The established term for this type of mechanism is ``payment channels''.

  As of today there exists no definitive systematisation of what payment channels can
  achieve and at what cost. Each proposal provides different benefits and incurs different
  costs; the language and formalism of each proposal is largely incompatible with the
  rest. Most of the proposals lack formal security definitions and proofs. More
  importantly, it is unclear what are the exact inherent tradeoffs and whether an ideal
  balance exists. One of the targets of this research is to provide a suitable conceptual
  framework that encompasses the whole range of possibilities and limitations of such
  channels.

  Two related issues that have not yet been tackled are the following. Firstly, currently
  all proposals offer trustless operation as a feature. Nevertheless, it seems plausible
  that a relaxation of this constraint may indeed allow for more efficient, faster and
  lighter payment channels. This is a new direction that this research will attempt to
  explore, ideally incorporating it in the general framework proposed above.

  Lastly, current blockchains do not focus on how to facilitate payment channels, instead
  attempt to obviate their need. An interesting topic would be to provide a treatment on
  what properties should a blockchain have to aid the creation of lightweight payment
  channels.

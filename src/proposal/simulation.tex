\subsection{Simulation--based Security}
  The two distinct research directions will both leverage tools and methods used in
  Cryptoghraphy and Game Theory. In order to formally model the interactions between
  marketplace participants, we will employ simulation--based security~\cite{simulate}
  and more specifically the Universally Composable Security framework~\cite{uc}. The
  general strategy of simulation--based security is as follows: First we give some
  security definitions that formalise the desirable properties our system should have.
  Subsequently we give an ideal--world functionality that satisfies these properties.
  Afterwards we describe a protocol that hopefully realises this functionality the
  functionality is realised if an external observer cannot tell whether she interacts
  with dummy parties that simply forward every message to and from the ideal
  functionality or with parties executing the protocol. Finally we prove that the
  proposed protocol indeed realises the functionality.

  The Universally Composable Security framework is a particular way of setting up and
  proving the security of a protocol. Achieving simulation of a protocol in the
  Universal Composition setting ensures that the protocol can be executed concurrently
  with other protocols that are secure in the Universal Composition setting.
  Consequently, managing to prove the security of our protocol in this setting ensures
  that it can coexist with other secure protocols in the same processing unit without
  suffering diminished security itself or damaging the security of the rest of the
  aforementioned protocols. Such an achievement would ensure that our protocol is more
  readily implementable whilst maintaining a high standard of security and would obviate
  the need of further security proofs for the case when it is executed in an interleaved
  fashion with other secure protocols.

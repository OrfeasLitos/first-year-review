\section{Directions for further work and plan of action}
\label{sec:plan}
  In this section we lay out our plan for further development of our models, known and
  possible hurdles we will have to overcome and a strategy to reroute our research to new,
  related topics in case the current approach does not bear fruit. We give a rough
  schedule of our projected progress at each time frame, along with the incremental
  results that we hope to lead to relevant publications.

  We will treat separately the e-commerce and the payment channels settings, starting from
  the simpler issues and gradually making our way to the most complex (and most
  interesting) ones for each setting.

  \subimport{./plan/}{ecommerceplan.tex}
  \subimport{./plan/}{channelsplan.tex}
  \begin{itemize}
    \item Missing pieces:
    \begin{itemize}
      \item What does the Adversary do?
      \item How can F\_Trust be substituted by Trust is Risk? Is a wrapper around Trust is
      Risk needed? Roughly describe protocol and possible implications
      \item Philosophical: Accept that there exist situations where consensus is
      unreachable and design for robustness and disincentives against it
      \item Philosophical: Opennes to different approaches towards money: Money is much
      less needed if "everybody knows everybody", because it is much more difficult to
      cheat. Thus a ledger that just contains the history of trades may obviate entirely
      the function of money. Such results have a weird compatibility status with results
      such as~\cite{marketequilibrium}.
    \end{itemize}
    \item Future expansions:
    \begin{itemize}
      \item arbitrary digital assets
      \item connect with real G\_Ledger
      \item decide whether and how to realize G\_Assets
      \item allow dishonest seller to send alternative asset as well
    \end{itemize}
  \end{itemize}

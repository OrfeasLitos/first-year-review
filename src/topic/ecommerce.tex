\subsection{Decentralised E-Commerce}
  Going in more detail, blockchains provide us with novel capabilities. It is possible to
  immutably and truthfully record the fact that Alice paid Bob some cryptographic coins on
  a blockchain; it is possible for Bob to provably and irrefutably give her a digital
  asset in return (e.g. an ICO~\cite{eos}, or a sword in a future blockchain--backed game)
  and record the fact on a blockchain as well; additionally, it is possible to make the
  exchange atomic so that no trust between the two parties is needed. Unfortunately, all
  the above are impossible to accomplish if the asset is physical. Alice may pay first and
  then Bob can refuse to give the asset; or Bob may indeed hand in the asset, but Alice
  may then claim that it was not as advertised and demand her money back. The possibility
  of fraud is exacerbated if the exhange takes place online between parties who do not
  know each other personally. Crucially, an external observer cannot always resolutely
  decide which of the parties is right. Even if she can, it seems impossible for her to
  prove that she is not colluding with either player and that she is completely fair and
  impartial.

  For comparison, if the payment is done in fiat currency, then there exists a mediator
  (usually a bank, often a court) who is charged with deciding what is the fair way to
  settle the dispute. Additionally, it is in the mediator's direct benefit to strive for
  the objectively fairest solution. Obviously the cryptographic security of this state of
  affairs is very low: it is entirely possible that the bank employee or the judge who has
  the final say is a friend of the fraudster. Nevertheless, it seems that currently the
  incentives are much better aligned when relying on the traditional legal system. As a
  result, transacting in fiat currency is, for practical and everyday purchases of
  physical goods and especially in the online setting, safer and more robust than using
  cryptocurrencies.

  The cryptographic achievements of blockchains are currently realised only in a closed
  system, where everything can be proven mathematically. Nevertheless, the economy is not
  such a system. In order for decentralised blockchains to substitute the traditional fiat
  currencies, a mechanism for decentralised e-commerce with guarantees of minimal trust
  is needed. We propose to attempt the creation of such a mechanism or give plausible
  arguments why such a mechanism is unattainable. In case we manage to create such a
  system, it should ideally be able to constitute the backend of a decentralised ecommerce
  platform, similar in usage to existing platforms such as OpenBazaar~\cite{openbazaar}.

  In essence, such a mechanism would provide a decentralised reputation system. In current
  online marketplaces (e.g. ebay.com) there exists a central authority that aggregates
  ratings and reviews from past interactions between users and serves them to participants
  of future transactions with the aim of minimising fraud. We aspire to replace this
  central authority with a distributed protocol or prove that such an endeavour is
  impossible.

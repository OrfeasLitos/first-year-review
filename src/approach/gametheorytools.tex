\subsection{Game Theoretic tools}
  This is where rational analysis and mechanism design~\cite{nisan2007algorithmic} come
  into play. Game Theory~\cite{leyton2008essentials} provides a series of tools and
  concepts that greatly support the study of multi-agent rational systems. The concept of
  Nash Equilibrium~\cite{nasheqcomp}, fundamental to the study of Game Theory, will be
  used throughout this research. Indeed, our aim will be to describe a mechanism that
  resembles the real e-commerce setting as closely as possible --- while abstracting away
  irrelevant complications --- and find strategies that constitute Nash Equilibria.

  General Bayesian games of incomplete information constitute another related powerful
  tool that will help us model the interactions between buyers and sellers. More
  specifically, the concepts of type, beliefs and knowledge will help us define the most
  beneficial action for each agent, within some computational constraints.

  Fortunately, we can leverage an arsenal of cryptographic tools to help us rule out some
  categories of undesired behaviours and thus design a system that achieves the goals of
  modern e-commerce whilst minimising the trust towards third parties. The guarantees
  given will not be necessarily cryptographic, but rather rational; that is, participants
  will refrain from acting badly not because they do not have the ability, but rather
  because it will be detrimental to their social standing, access to goods and services or
  other desirable attributes. We hope that this combination of Game Theoretic analysis and
  Cryptography will prove powerful enough to formalise provably secure and strategy-proof
  systems for use by decentralised marketplaces.

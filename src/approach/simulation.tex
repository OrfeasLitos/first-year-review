\subsection{Simulation--based Security}
  The two distinct research directions will both leverage tools and methods used in
  Cryptoghraphy and Game Theory. In order to formally model the interactions between
  marketplace participants, we will employ simulation--based security~\cite{simulate}
  and more specifically the Universally Composable Security framework~\cite{uc}. The
  general strategy of simulation--based security is as follows: First we give some
  security definitions that formalise the desired properties our system should have.
  Subsequently we give an ideal--world functionality that satisfies these properties.
  This functionality acts as a centralised trusted party that executes the desired
  operation on behalf of the participants. Afterwards we describe a protocol that
  hopefully realises this functionality. Finally we prove that the proposed protocol
  indeed realises the functionality by showing that an external observer cannot
  distinguish an execution of the functionality from an execution of the actual
  protocol.

  To make matters worse (i.e. closer to the real-world challenge the cryptographic
  protocol faces), the concept of the Adversary is introduced. This party is in control
  of all messages that the honest participants exchange and can change, delay or
  completely block them at will. A complete security proof of the fact that a given
  protocol realises a particular functionality states that for every efficient Adversary
  (polynomial algorithm) there exists a Simulator (polynomial algorithm) such that the
  distributions of the execution of the functionality as viewed from the perspective of
  each participant are statistically close to the respective distributions of the
  execution of the protocol.

  The Universally Composable Security framework is a particular way of setting up and
  proving the security of a protocol. Achieving simulation of a protocol in the
  Universal Composition setting ensures that the protocol can be executed concurrently
  with other protocols that are secure in the Universal Composition setting.
  Consequently, managing to prove the security of our protocol in this setting ensures
  that it can coexist with other secure protocols in the same processing unit without
  suffering diminished security itself or damaging the security of the rest of the
  aforementioned protocols. Such an achievement would ensure that our protocol is more
  readily implementable whilst maintaining a high standard of security and would obviate
  the need of further security proofs for the case when it is executed in an interleaved
  fashion with other secure protocols.

\subsection{Need for Game Theoretic Analysis}
  Before the advent of blockchains, most cryptographic protocols solved problems where
  there were clear-cut lines between the aims of honest parties and the machinations of
  the Adversary. For example, in the case of public key encryption~\cite{dh}, the honest
  parties should be able to exchange messages encrypted in a fashion that only those who
  know the private key can decrypt, even though the adversary has the ability to employ
  any method conceivable (within some computational limits) to read the original contents
  of the message. Any protocol that claims to realise secure encryption should satisfy
  (the formal version of) this requirement. It is intuitively obvious that it is in the
  benefit of the honest parties to follow such a protocol.

  On the other hand, blockchains made the state of affairs much more complex. Indeed,
  bitcoin was a concrete solution to an not yet defined problem; several years passed
  before a formal definition of the problem blockchains solve was
  formulated~\cite{backbone}. Nevertheless, this definition does not necessarily coincide
  with the desires of the participants to the protocol. The definitions demand that a
  blockchain realises an immutable ledger that is constantly extended, however the goals
  of the participants span a great variety, such as maximising their share of coins or
  preventing certain transactions from entering the blockchain. In this case, assuming
  that there exist only honest parties that execute the protocol exactly as described and
  an Adversary that employs every conceivable technique in order to break the security
  definitions is an oversimplification. It ignores slight but crucial deviations from the
  protocol (e.g.~\cite{selfishmine}) that may provide advantages not considered in the
  security definitions. Therefore the assumption of an honest majority can be contested.

  When considering decentralised e-commerce, the problem is even more exacerbated.
  Formulating a simple definition for the characteristics of a healthy marketplace is
  highly non-trivial. The individual desires of parties should employ a central role in
  the analysis and not be necessarily restricted to a clear-cut protocol. Assuming that a
  portion of honest participants exists may be wrong, especially if other participants
  follow counter-strategies that make honest behaviour unprofitable or if honest players
  have incentive to profitably deviate from their strategy. Conversely, permitting
  arbitrary actions to the Adversary could complicate the analysis more than necessary,
  since it may be reasonable to assume that certain ``unreasonable'' strategies will never
  be followed by the Adversary.
